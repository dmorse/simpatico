%%%%%%%%%%%%%%%%%%%%%%%%%%%%%%%%%%%%%%%%%
% Simple Sectioned Essay Template
% LaTeX Template
%
% This template has been downloaded from:
% http://www.latextemplates.com
%%%%%%%%%%%%%%%%%%%%%%%%%%%%%%%%%%%%%%%%%

%----------------------------------------------------------------------------------------
%	PACKAGES AND OTHER DOCUMENT CONFIGURATIONS
%----------------------------------------------------------------------------------------

\documentclass[12pt]{article} % Default font size is 12pt, it can be changed here

%\usepackage{geometry} % Required to change the page size to A4
%\geometry{a4paper} % Set the page size to be A4 as opposed to the default US Letter

\usepackage{graphicx} % Required for including pictures
\usepackage{float} % Allows putting an [H] in \begin{figure} to specify exact location of figure
\usepackage{wrapfig} % Allows in-line images such as the example fish picture

%\linespread{1.2} % Line spacing
%\setlength\parindent{0pt} % Uncomment to remove all indentation from paragraphs
%\graphicspath{{./Pictures/}} % Specifies the directory where pictures are stored

\newcommand{\eg}{{\it e.g.}}
\newcommand{\vv}[1]{\mathbf{#1}}

\newcommand\system{\Gamma^{N}}
\newcommand\chain{\Gamma}

\newcommand\Utot{U_{N+1}}
\newcommand\Usys{U_{N}}
\newcommand\delU{\Delta U}
\newcommand\Uid{U^{({\rm id})}}
\newcommand\Uext{U^{({\rm ext})}}

\newcommand\Ztot{Z_{N+1}}
\newcommand\Zsystem{Z_{N}}
\newcommand\Zid{Z^{({\rm id})}}

\newcommand\Ftot{F_{N+1}}
\newcommand\Fsystem{F_{N}}
\newcommand\Fid{F^{({\rm id})}}

\newcommand\muid{\mu^{({\rm id})}}
\newcommand\muex{\mu^{({\rm ex})}}

\newcommand\vbead{\vv{r}}
\newcommand\qbead{\vv{r}_{1}}
\newcommand\nbead{l}

\newcommand\vbond{\vv{b}}
\newcommand\ubond{\vv{u}}
\newcommand\rbond{b}
\newcommand\nbond{l-1}

\newcommand\ntrial{k}
\newcommand\rtrial{\tilde{b}}
\newcommand\vtrial{\tilde{\vv{b}}}
\newcommand\utrial{\tilde{\vv{u}}}

\newcommand\Ubond{U^{({\rm bond})}}
\newcommand\Uornt{U^{({\rm ornt})}}
\newcommand\Uangle{U^{({\rm angle})}}
\newcommand\Udihedral{U^{({\rm dihedral})}}
\newcommand\Pbond{P^{({\rm bond})}}
\newcommand\Pornt{P^{({\rm ornt})}}
\newcommand\zid{z^{({\rm id})}}

\newcommand\rconfig{\tilde{\chain}}
\begin{document}

%-------------------------------------------------------------------------------
%	TITLE PAGE
%-------------------------------------------------------------------------------

\begin{titlepage}

\newcommand{\HRule}{\rule{\linewidth}{0.5mm}} % Defines a new command for the horizontal lines, change thickness here

\center % Center everything on the page

\textsc{\LARGE University of Minnesota}\\[1.5cm] % Name of your university/college
\textsc{\Large Simpatico}\\[0.5cm] % Major heading such as course name
\textsc{\large Algorithm Notes}\\[0.5cm] % Minor heading such as course title

\HRule \\[0.4cm]
{ \huge \bfseries Bias Insertion and Sampling Algorithms}\\[0.4cm] % Title of your document
\HRule \\[1.5cm]

\begin{minipage}{0.4\textwidth}
\begin{flushleft} \large
\emph{Author:}\\
Taher \textsc{Ghasimakbari} % Your name
\end{flushleft}
\end{minipage}
~
\begin{minipage}{0.4\textwidth}
\begin{flushright} \large
\emph{Supervisor:} \\
David \textsc{Morse} % Supervisor's Name
\end{flushright}
\end{minipage}\\[4cm]

{\large \today}\\[3cm] % Date, change the \today to a set date if you want to be precise

%\includegraphics{Logo}\\[1cm] % Include a department/university logo - this will require the graphicx package

\vfill % Fill the rest of the page with whitespace

\end{titlepage}

%----------------------------------------------------------------------------------------
%	TABLE OF CONTENTS
%----------------------------------------------------------------------------------------

%\tableofcontents % Include a table of contents

%\newpage % Begins the essay on a new page instead of on the same page as the table of contents 

%----------------------------------------------------------------------------------------
%	INTRODUCTION
%----------------------------------------------------------------------------------------

\section{Chemical Potentials and Insertion}
Consider a system that initially contains $N$ molecules of some species $S$ of interest, to which we will try to add one more molecule of species $S$. The system may contain any number of molecules of any number of other species. Let $\system$ denote the coordinates of all molecules of all species that are present in the initial state, before addition of the $(N+1)$th molecule of type $S$. Let $\chain$ denote a set of coordinates required to specify the spatial configuration of the added molecule. The potential energy $\Utot$ of a system with this added $S$ molecule can be written as a sum 
\begin{equation}
    \Utot(\system, \chain) = \Usys(\system) + \delU (\chain, \system) 
    \quad.
\end{equation}
Here, $\delU$ is the potential energy change arising from insertion of the molecule with a known conformation $\chain$ into a system with a known configuration $\system$ for all other molecules. The change $\delU$ generally includes the intramolecular covalent energy of the added molecule and all interm- and intra-molecular nonbonded pair interactions that involve the added molecule. 

Let $\Zsystem$ and $\Ztot$ denote the partition functions of a system without the added molecule and with an added $S$ molecule, respectively. These are given by
\begin{eqnarray}
    \Zsystem & \equiv & \frac{C}{N!}\int d\system e^{-\beta \Usys(\system)} \\
    \Ztot  & \equiv & \frac{C}{(N+1)!}\int d\system e^{-\beta \Usys(\system)} 
                       \int d\chain e^{-\beta \delU(\chain, \system)}
    \quad.
\end{eqnarray}
where $\beta = 1/(k_{B}T$, where $k_{B}$ is Boltzmann's constant.  Here, the constant $C$ represents any product of combinatorical (inverse factorial) factors arising from species other than species $S$.  Let $\Fsystem = -k_{B}T\ln \Zsystem$ and $\Ftot = -k_{B}T\ln \Ztot$ denote the corresponding free energies.  The chemical potential of species $S$ is given by the free energy difference
\begin{eqnarray}
   \mu & = & \Ftot - \Fsystem \nonumber \\
       & = & -k_{B}T \ln \left ( \frac{\Ztot}{\Zsystem (N+1)} \right )
      \label{mudef}
\end{eqnarray}
The required ratio $\Ztot/\Zsystem$ can be rewritten as an average
\begin{equation}
  \frac{\Ztot}{\Zsystem} = 
  \left \langle \int d\chain e^{-\beta \delU(\chain, \system)} \right \rangle_{N}
\end{equation}
in which 
\begin{equation}
  \langle \cdots \rangle_{N} \equiv
  \frac{ \int d\system e^{-\beta \Usys(\system)} \cdots }
       { \int d\system e^{-\beta \Usys(\system)}}
\end{equation}
denotes an average of some quantity ``$\cdots$" over the thermal equilibrium 
distribution for the system without the added molecule. 

For some purposes, it is useful to divide $\delU$ into a sum
\begin{equation}
   \delU(\system, \chain) = \Uid(\chain) + \Uext (\chain, \system)
\end{equation}
of a ``ideal" bonded contribution $\Uid(\chain)$ that depends only on the conformation of the added molecule and an "external" contribution $\Uext(\chain,\system)$ that depends upon the configuration of both the added molecule and the original system.  There is some freedom in how this division is defined, \eg, the ``ideal" Hamiltonian may be taken to include or exclude non-bonded pair interactions among particles within the added test molecule. Let 
\begin{equation}
    \Zid_{1} \equiv \int d\chain \l e^{-\beta \Uid(\chain) }
\end{equation}
be the partition function for system containing a single $S$ molecule with potential energy $\Uid$, confined to the same volume as that of the entire interacting system of interest. The chemical potential for an ``ideal" reference system of $N$ such non-interacting molecules, defined by analogy to Eq. (\ref{mudef}), is then
\begin{eqnarray}
    \muid & \equiv & \Fid_{N+1} - \Fid_{N}  \nonumber \\
          & =      & -k_{B}T\ln(\Zid_{1}/N+1)
    \quad.
\end{eqnarray}
We define an excess chemical potential $\muex$ for species $S$ as the difference
\begin{equation}
     \muex \equiv \mu - \muid
\end{equation}
between the chemical potentials of the real and ideal systems containing the same number of molecules in the same total volume. This is given by a dimensionless ratio
\begin{equation}
   \muex_{N}  = -k_{B}T \ln \left ( \frac{\Ztot}{\Zsystem \Zid_{1}} \right )
\end{equation}
where
\begin{equation}
  \frac{\Ztot}{\Zsystem \Zid_{1}} = 
  \frac{\left \langle \int d\chain e^{-\beta \delU(\chain, \system)} \right \rangle_{N}} 
       {\Zid_{1}} 
  \quad.
\end{equation}
This is the quantity that is calculated in insertion methods.

\section{Linear molecules}
We will focus in what follows on algorithm that inserting a linear polymer chain 
by "growing" the chain one atom at a time. Here we introduce some notation for 
chain molecules that is useful in the analysis of the Rosenbluth algorithm.

Consider a linear chain of $\nbead$ point like particles with positions 
$\vbead_{1},\ldots,\vbead_{\nbead}$ connected by $\nbond$ bonds. Let 
\begin{equation}
   \vbond_{i} = \rbond_{i} \ubond_{i} = \vbead_{i} - \vbead_{i-1}
\end{equation}
denote the bond vector between beads $i$ and $i-1$, in which 
$\rbond_{i} \equiv |\vbond_{i}|$ is the bond length and and $\ubond_{i}$ is a unit 
vector.  

In what follows, we will consider and algorithm (Rosenbluth sampling) in which a chain is grown sequentially by first choosing a position of bead 1, and then for 2 bead, and and so on, until the chain is complete.  The configuration $\system$ of all neighboring molecules is kept fixed throughout this growth process. During this hypothetical growth process, for a given choice of the system configurations $\system$ and a given choice of the positions $\vbead, \ldots, \vbead_{i-1}$, let $\delU_{i}(\vbead_{i})$ denote the change in potential energy associated with addition of bead $i$.  This quantity will be referred to as the incremental energy for bead i. It includes all contributions to $\delU$ arising from bonded and nonbonded interactions involving bead $i$, including interactions with beads $1,\ldots, i-1$ and with surrounding chains, but excluding any bonded or nonbonded interactions involving beads $i+1,\ldots,\nbead$, which do not yet exist. By definition, for a given final chain configuration $\chain$,
\begin{equation}
   \delU(\chain, \system) = \sum_{i=1}^{\nbead} \delU_{i} \quad.
\end{equation}
If $\delU$ is divided into an ``ideal" and ``external" part, then we may define analogous ideal and external incremental energies $\Uid_{i}$ and $\Uext_{i}$. The ideal incremental contribution $\Uid_{i}$ can only depend on the positions of beads $1,\ldots,i$ of the added chain, while the external incremental contribution can depend on these bead positions on on the configuration $\system$ of the surrounding molecules.

In what follows, a model for the ``ideal" potential energy will be said to be ``separable" if the ideal single-chain partition function can be expressed as a product
\begin{equation}
    \Zid_{1} = V \prod_{i=2}^{\nbead} \zid_{i}
    \label{ZidProduct}
\end{equation}
in which $V$ is the volume of the system, and in which
\begin{equation}
    \zid_{i} \equiv \int d\vbond 
    \exp( -\beta \Uid_{i}(\rbond, \ubond) )
    \label{zid}
\end{equation}
is an ideal partition function factor associated with addition of a particle or bond $i$ to a chain with known positions for beads $1, \ldots, i-1$. Because the overall ideal chain partition function $\Zid_{1}$ is not a function of the bead postiions, eq. (\ref{ZiProduct}) can be valid only if $z_{i}$ is independent of the positions of beads $i, \ldots, i-1$.

The standard molecular mechanics formula for the bonded (covalent) energy of a chain molecule is separable in the above sense. In this model, the incremental energy for all beads $i > $ is given by a sum
\begin{equation}
   \Uid_{i} = \Ubond_{i}(\rbond_{i}) + \Uornt_{i}(\ubond_{i})
\end{equation}
in which $\Ubond_{i}$ is a bond energy that depends only on the length of bond $i$, and 
$\Uornt_{i}(\ubond)$ is an orientation depend energy that depends only on the orientation 
of $\ubond$ of bond $i$ relative to the orientation of bond $i-1$ and relative to the plane 
defined by bonds $i-2$ and $i-1$. Specifically, this standard model expresses 
$\Uornt_{i}(\ubond)$ for all $i > 3$ as a sum
\begin{equation}
   \Uornt_{i}(\ubond_{i}) = \Uangle(\theta_{i}) + \Udihedral(\phi_{i}) 
\end{equation}
of an bending angle component $\Uangle(\theta)$ that (for $i>2$) depends only on the angle $\theta$ between bonds $i-1$ and $i$ and a dihedral compoennt $\Udihedral(\phi)$ that (for $i > 3$) depends only on the dihedral angle between bond $i$ and the plane defined by bonds $i-2$ and $i-1$. The dihedral component is zero for bonds $2$ and $3$ (since a dihedral potential involves three bonds), and the angle component is zero for bond $3$. By writing the integral over bond vectors in Eq. (\ref{zid}) in terms of local polar coordinates $b_{i}, \theta_{i}, \phi_{i}$ for bond $i$, using a frame of reference defined by the orientation of the previous two bonds, it is straightforward to show the value of $z_{i}$ for each bond is independent of the choices made for bonds $i=2,\ldots,i-1$, and that the ideal chain partition function is separable in the sense defined by Eq. (\ref{ZidProduct}).

At each stage of growth of a chain with an ideal potential energy $\Uid$ given by the above standard, we may also define an conditional equilibrium distribution for the bond length $\rbond_{i}$ and $\ubond_{i}$. The ideal equilibrium distribution for the bond length is given, to within a normalization constant,
\begin{equation}
    \Pbond(\rbond_{i}) \propto \rbond^{2} \exp( -\beta \Ubond_{i}(\rbond_{i}) )
\end{equation}
The factor of $\rbond^{2}$ is required to account for the volume element in polar coordinates. 
The ideal equilibrium distribution for unit vector $\ubond_{i}$, given knowledge of orientations for bonds $i-1$ and $i-2$, is given by
\begin{equation}
    \Pornt(\ubond) \propto \exp( -\beta \Uornt_{i}(\ubond))
    \quad.
\end{equation}
The overall normalized ideal probability distributions is given by
\begin{equation}
   \Pbond(\rbond)\Pornt(\ubond) = \frac{e^{-\beta\Uid_{i}(\vbond) }} {\zid}
   \label{PtrialNormalized}
\end{equation}
where $\zid$ is defined in Eq. (\ref{zid}.

\section{Rosenbluth Sampling} % Sub-section
Rosenbluth sampling is a method of "growing" an ensemble of low energy conformations for linear chains, which is the basis of several closely related algorithms for measuring chemical potentials and for conformational sampling. The method generates an ensemble of conformations that is not the equilibrium ensemble, but that can be reweighted in order to reproduce an equilibrium ensemble. The method was originally developed on a lattice and then generalized to continuum models. The version we describe here is for linear bead-spring chains with a potential energy that has an "ideal" part that is described by the standard molecular mechanics model described above. The ideal potential energy must have a bond contribution, and may have angle and dihedral contributions of the form described above. We use the notation of theprevious section for chain conformations and incremental free energies, and "ideal" distributions for bond lengths and orientations.

The Rosenbluth algorithm for growing a trial chain conformation, in outline, works as 
follows:
\begin{itemize}
\item The position $\qbead$ of the first bead, is chosen at random within the system 
      volume.
\item For each subsequent bead $i=2,\ldots, \nbead$, we choose $\ntrial$ trial 
positions or, equivalently, $\ntrial$ trial values of the bond vector $\vbond_{i}$. 
Let $\vtrial_{i}^{j}$ denote the $j$th trial bond vector for bond $i$.
\item For each bond, one of the $\ntrial$ trial bond vectors is then selected with a 
probability criterion that tends to favor trials with low values of $\delU^{ext}_{i}$.
Let $\vbond_{i}$ denote the selected bond vector for bond $i$. 
\end{itemize} 
One specfiic set of rules for generating and selecting bond vectors is discussed below. 
The construction of bonds is repeated until the chain is complete.

For each bond $i$, a set of $\ntrial$ trial bond vectors is generated as follows.
First a single bond length $\rbond_{i}$ is chosen from the ideal equilibrium 
distribution $\Pbond_{i}(\rbond_{i})$. Then, a set of $\ntrial$ unit vectors 
$\utrial_{i}^{j}$ with $j=1,\ldots,\ntrial$ are chosen from probability distribution 
$\Pornt_{i}(\utrial)$.  This generates a set of $\ntrial$ trial bond vectors with 
the same bond length but different orientations.  One of these trial bond vectors 
is then selected with a probability $S^{j}_{i}$ given by
\begin{equation}
    S^{j}_{i} = \frac{e^{-\beta\Uext_{i}(\vtrial^{j}_{i})}}{w_{i}}
    \label{PtrialSelection},
\end{equation}
in which 
\begin{equation}
     w_{i} \equiv \sum_{j=1}^{\ntrial} e^{ -\beta \Uext_{i}(\vtrial^{j}_{i})}
     \label{RosenbluthBond}
\end{equation}
is the so called Rosenbluth weight associated with bond $i$.

In what follows we will refer to the overall result of this growth process, 
including the discarded trial bond orientations, as a Rosenbluth configuration. A 
Rosenbluth configuration is defined by a choice for the position of the first bead, 
a choice of a bond length $b_{i}$ for each bond $i$, a set of $\ntrial$ trial 
orientations $\utrial_{i}^{j}$ for each bond, and choice $\vbond_{i}$ of which 
trial bond vector is selected for each bond.  Let $s_{i}$ denote the index of 
the selected trial orientation for bond $i$, so that 
$\vbond_{i} = \vtrial_{i}^{s_{i}}$.  Let $\rconfig$ denote the entire Rosenbluth 
configuration generated by this process, which is given by the variables
\begin{equation}
   \rconfig = \left \{ \vbead_{1},\rbond_{i},\utrial_{i}^{j},s_{i} \right \}
\end{equation}
for all $i=2,\ldots,\nbead$ and $j=1,\ldots,\ntrial$. The relationship between 
the Rosenbluth configuration $\rconfig$ and the final chain configuration $\chain$ 
is many-to-one: Each Rosenbluth configuration contains a single final configuration, 
but there are many possible Rosenbluth configurations that could lead to the same 
final configuration.  The probability for a Rosenbluth configuration is given by 
a product of
\begin{equation}
   P(\rconfig) = \frac{1}{V}
                 \prod_{i=2}^{\nbead} \left (
                 S_{s_{i}}^{i}
                 \Pbond(\rbond_{i})
                 \prod_{k=1}^{\ntrial}
                 \Pornt(\utrial_{i}^{k}) \right )
   \quad,
\end{equation}
where $P_{i}^{j}$ is given by Eq. (\ref{PtrialSelection}), and where $\Pbond$
and $\Pornt$ are normalized ideal equilibrium distributions.  The factor of 
$1/V$ arises from the choice of the position $\vbead_{1}$ of the first bead. 

Let $\int d\rconfig$ denote an integral over all possible Rosenbluth configuration. 
This implicitly involves a sum over possible choices for the index $s_{i}$ of the 
selected orientation for each bond, as well as integration over bond lengths and 
trial bond orientations, and so may be written as
\begin{equation}
    \int d\rconfig \equiv 
                   \int d\qbead
                   \prod_{i=2}^{\nbead} 
                   \sum_{s_{i}=1}^{\ntrial} 
                   \int d\rbond_{i}
                   \prod_{j=1}^{\ntrial}
                   \int d\vbond_{i}^{j}
                   \quad.
\end{equation}
It is straightforward to confirm that $\int d\rconfig P(\rconfig) = 1$.

We define a Rosenbluth weight $W(\rconfig)$ associated with each Rosenbluth 
configuration $\rconfig$ as a product
\begin{equation}
   W(\rconfig) = w_{1} \prod_{i=2}^{\nbead} \frac{w_{i}}{k}
\end{equation}
where $w_{i}$ is Rosenbluth weight for bond $i$, defined in Eq. (\ref{RosenbluthBond}),
and in which
\begin{equation}
   w_{1} = e^{-\Uext_{1}(\vbead_{1})} 
\end{equation}
is a Rosenbluth factor for the first bead. It is straightforward to show that: 
\begin{equation}
   P(\rconfig)W(\rconfig) 
   = \frac{1}{V} e^{-\beta\Uext(\chain)}
                 \prod_{i=2}^{\nbead} \left (
                 \frac{1}{k}
                 \Pbond(\rbond_{i})
                 \prod_{k=1}^{\ntrial}
                 \Pornt(\utrial_{i}^{k}) \right )
   \label{PWRosenbluthProduct}
\end{equation}
where $\Uext(\chain)$ is the total external potential of the added chain.

{\it Theorem}: Let $A(\Gamma)$ be any function that depends only on the final 
configuration, and not on the trials. We may show that for any such function,
\begin{equation}
    \int d\rconfig P(\rconfig)W(\rconfig) A(\chain) = 
    \frac{\int d\chain e^{-\beta\delU(\chain)}A(\chain)}
         {\Zid} 
    \label{RosenbluthIntegralTheorem}
\end{equation}
\vspace{12pt}

Proof: Using Eq. (\ref{PWRosenbluthProduct}), we find that
\begin{equation}
   \int d\rconfig P(\rconfig)W(\rconfig) A(\chain)
   = \frac{1}{V} \int d\rconfig 
    \prod_{i=2}^{\nbead} \left (
     \frac{1}{k} e^{-\beta\Uext_{i}(\vbond_{i})}
                 \Pbond(\rbond_{i})
                 \prod_{k=1}^{\ntrial}
                 \Pornt(\utrial_{i}^{k}) \right )
\end{equation}
We can do the sum over values of the index of the selected trial orientation
for each bond integrals over rejected trial orientations iteratively: We first
considering the sum over values of the index $s_{i}$ and over rejected trial
orientations for the last bond, with $i=\nbead$, then apply the same reasoning 
to bond $\nbead-1$, etc. The factor arising from summation over possible values 
of the selected trial index $s_{i}$ and integration over trial configurations 
for bond $i$ is of the form
\begin{equation}
   \frac{1}{k} \sum_{s_{i}=1}^{\ntrial}
   \prod_{j=1}^{\ntrial} \int d\utrial_{i}^{k} 
   \Pornt(\utrial_{i}^{k}) A(\chain) 
   \frac{1}{k} e^{-\beta\Uext_{i}(\vbond_{i})}
   \quad.
\end{equation}
The sum over possible values of $s_{i}$ generates $k$ equivalent integrals.
Each such integral involves one integral over the chosen trial orientation
and $k-1$ integrals over the remaining trials, in which the functions $\Uext$ 
and $A$ depend only on the chosen trial orientation. In each such integral,
each rejected trial orientation enters the integrand only through the 
associated factor of $\Pornt$. Because $\Pornt$ is a normalized distribution, 
each integral over a rejected trial orientation yields of unity. By dividing
the sum of $k$ identical integrals by $k$, we thus obtain reduce the above
expression to an integral
\begin{equation}
   \frac{1}{k} \int d\utrial_{i} \Pornt(\utrial_{i}) A(\chain) 
   \frac{1}{k} e^{-\beta\Uext_{i}(\vbond_{i})}
\end{equation}
over the orientation of the selected trial, with no reference to the rejected 
trials. By applying this to each bond in sequence, and thus removing all 
integrals over rejected bond orientations, we obtain an integral involving only 
integrals over the selected bond lengths and orientations. The result may be
expressed as an integral
\begin{equation}
    \int d\rconfig P(\rconfig)W(\rconfig) A(\chain) = 
    \int d\chain \frac{1}{V} e^{-\beta\Uext(\chain)}
                 \prod_{i=2}^{\nbead} \left ( \Pbond(\rbond_{i}) \Pornt(\ubond_{i}) \right )
                 A(\chain) \quad.
\end{equation}
over the selected chain configuration $d\chain$. Using Eq. (\ref{PtrialNormalized})
for the normalized equilibrium distribution then yields
\begin{equation}
    \int d\rconfig P(\rconfig)W(\rconfig) A(\chain) = 
    \int d\chain \frac{e^{-\beta\delU(\chain)}A(\chain)}{ V\prod_{i=2}^{\nbead}\zid_{i}}
\end{equation}
for any function $A(\chain)$ of the chain conformation. For any separable model,
we may then use Eq.  (\ref{ZidProduct}) to obtain Eq. (\ref{RosenbluthIntegralThm}).
(QED)

In order to relate the Rosenbluth factor to the chemical potential, we set $A(\Gamma)=1$
in the above theorem to show that
\begin{equation}
    \langle W \rangle_{\rconfig} = \int d\rconfig P(\rconfig)W(\rconfig) = 
    \frac{\int d\chain e^{-\beta\delU(\chain)}}{\Zid} 
    \quad,
\end{equation}
where $\langle \cdots \rangle_{\rconfig}$ denotes an average over Rosenbluth configurations.
Taking the average of this quantity respect to configuration $\Gamma$ of the system without 
an added chain then yields an expression for the excess activity $e^{\beta\muex}$ as an 
average of the Rosenbluth weight
\begin{equation}
   e^{\beta \muex} = 
   \frac{\Ztot}{\Zsystem\Zid} = 
   \langle \langle W \rangle_{\rconfig} \rangle
\end{equation}
over both Rosenbluth configurations and equilibrium solvent chain configurations. 

%----------------------------------------------------------------------------------------
%	BIBLIOGRAPHY
%----------------------------------------------------------------------------------------

\begin{thebibliography}{99} % Bibliography - this is intentionally simple in this template

[1] D. Frenkel, B. Smit. Understanding Molecular Simulation. 
 
\end{thebibliography}

%----------------------------------------------------------------------------------------

\end{document}
